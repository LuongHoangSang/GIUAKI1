% --- tạo đường kẻ chấm 
	\newcommand{\dhorline}[2]{\hbox to #1{\leaders\hbox to 0.5em{\hss . \hss}\hfil}}
	
% --- chừa chỗ trống tương ứng trong câu 
	
	\newlength\rulegoal
	\newlength\drule
	\setlength{\drule}{0.5em}
	\newcommand\breakableruleaux[2]{%
		\ifdim\rulegoal>\drule\relax%
		\dhorline{\drule}{#2}\allowbreak%
		\def\next{\breakableruleaux{#1}{#2}}%
		\else
		\dhorline{\rulegoal}{#2}%
		\def\next{}%
		\fi 
		\addtolength{\rulegoal}{-\drule}%
		\next%
	}
	
	\newcommand\breakablerule[2]{\setlength{\rulegoal}{#1}\breakableruleaux{#1}{#2}}
	
	\newlength{\myblanklength}
	\renewcommand{\bltext}[1]
	{
		~\hspace*{-7.5pt}\relax
		\settowidth{\myblanklength}{#1}
		\breakablerule{\the\myblanklength*\real{1.5}}{0.2pt}
	}
	
	\renewcommand{\xtrule}{\hspace*{-5pt}\dhorline{1cm}{0.2pt}~}
	
% --- % che công thức trên dòng riêng:
\renewcommand{\phantomeqn}[1]{
	%		\phantomline[b]{1} % dòng trắng
	\phantomline[d]{1} % dòng kẻ chấm
	%		\phantomline[l]{1} % dòng kẻ liền
}

% che đoạn văn bản 
	
	\renewcommand{\hide}[2][b]{
		\ifx b#1 \blankhide{#2} \else
		\ifx d#1 \dotshide{#2} \else
		\ifx l#1 \lineshide{#2}
		\fi\fi\fi 
	}

% che bài giải tất cả ví dụ 
%\usetikzlibrary{decorations.shapes}
%	\tikzset{

%		decoration = {shape backgrounds, shape=circle,

%			shape size=0.25pt, shape sep=2pt},

%		paint/.style = {decorate, fill=black}

%		}% tạo dạng dots 

	\renewcommand{\vidu}[3] % -- không đánh số, có lời giải 
		{
			%\vspace{0.3cm}
			\needspace{4\baselineskip}
			\noindent\textbf{Ví dụ \quad\ \mkstar{#1}}
			\begin{flushright}
				\leavevmode\vspace{-15pt}
				\begin{tcolorbox}[
					standard jigsaw
					,opacityback=0
					,opacityframe=0
					,width=0.95\textwidth
					,breakable
					,right=-4pt,top=-4pt,left=-4pt
					,colframe=white
					,colback=white
					,before upper={\parindent15pt}
					]
				
					{#2}
				\end{tcolorbox}
			
				\begin{tcolorbox}
					[enhanced
					,frame hidden
					,width=0.95\textwidth
					,borderline={0.3mm}{0.5mm}{dotted}
					,colback=white
					,title=Bài giải
					,fonttitle=\bfseries
					,coltitle=black
					,breakable
					,attach boxed title to top center={yshift=-\tcboxedtitleheight/2}
					,boxed title style={colback=green!10!white}]
					\hide{#3}
				\end{tcolorbox}	
				
			\end{flushright}	
		}
	
	\renewcommand{\viduii}[3] % có đánh số, có lời giải 
		{
			%\vspace{0.3cm}
			\refstepcounter{viduii}
			\needspace{4\baselineskip}
			\noindent\textbf{Ví dụ \theviduii\ \mkstar{#1}}
			\begin{flushright}
				\leavevmode\vspace{-15pt}
				\begin{tcolorbox}[
					standard jigsaw
					,opacityback=0
					,opacityframe=0
					,width=0.95\textwidth
					,breakable
					,right=-4pt,top=-4pt,left=-4pt
					,colframe=white
					,colback=white
					,before upper={\parindent15pt}
					]
					
					
					{#2}
				\end{tcolorbox}
					%\begin{center}
					%	\textbf{Giải:}
					%\end{center}
					
				\begin{tcolorbox}
					[enhanced
					,width=0.95\textwidth
					,frame hidden
					,borderline={0.3mm}{0.5mm}{dotted}
					,colback=white
					,title=Bài giải
					,fonttitle=\bfseries
					,coltitle=black
					,breakable
					,attach boxed title to top center={yshift=-\tcboxedtitleheight/2}
					,boxed title style={colback=green!10!white}]
					\hide{#3}
				\end{tcolorbox}	
			\end{flushright}	
		}
	